%\section{Conclusion and Future Work}\label{sec:conc}
\section{Conclusion}\label{sec:conc}
We have presented the most complete formal semantics of \ISA user-level instructions
to date, and have thoroughly tested it using synthesized test inputs and the GCC torture tests.
%in different formal analyses such as symbolic execution, deductive verification, and translation validation.
We have also illustrated several potential uses of the semantics which are realized by the formal analysis tools
derived right from the \K specification. 
%Using the \K framework for our specification automatically provides several formal analysis tools from the specification, which greatly simplified these applications.
The \K framework also enables us to represent a semantics as SMT theories,
which other projects can
leverage for their own purposes.
%We describe several practical lessons we have learned from our experience in developing the semantics, which could be useful for future formal specifications of processor ISAs.


%As mentioned in Section~\ref{sec:Approach:Overview}, we found that our ideas of extending \Strata do not scale because we need information about the instructions, whose semantics we want to learn, in order to constraint the search space. This lesson suggested a new idea: the information needed to reduce the search space can be automatically extracted from the manual. This  information does not need to be precise, and we believe that such rough information can be automatically extracted from the manual using text processing.  We plan to explore this idea while defining the semantics of unspecified and/or new instructions.

%\SC{We plan to test our model against various x86-64 implementations (like AMD), which could uncover flaws in those implementations and/or additional imperfections in the manual.}

%\SC{Also, we aim to use our semantics for translation validation of the entire LLVM compiler back-end for X86-64, i.e., hundreds of thousands of lines of C++ code. Scalability is achieved by the modularity of the translation validation technique, where the verification of (small) sub-components are verified individually (hence can be massively parallelized) and their results are combined to obtain the final claim about the entire system.}

 

%\K  has success stories about defining the formal semantics of production languages like C~\cite{KC} and LLVM~\cite{KLLVM}. We can use that towards the benefit of binary lifters like ~\cite{McSema:Recon14, FCD} which lifts the binary code to LLVM IR for binary analysis and/or optimization. Having the semantics of both the source (\ISA) and the target language(LLVM) help in verifying the translation using symbolic reasoning and hence enhance trust in  the translation.

%\textit{Acknowledgements.}
%We warmly thank the \K team
%   for their technical support throughout the project.  Also we thank the Strata
%   developers for promptly confirming
%   our reported bugs and for answering all our questions in great detail.
   
