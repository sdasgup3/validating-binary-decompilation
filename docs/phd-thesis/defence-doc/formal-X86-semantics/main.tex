\chapter{Formal Semantics of \ISA User-Level ISA}\label{sec:results}

The equivalence checker being parametric on the semantics of source (\ISA) and
target (LLVM) languages of decompilation, one of the important steps towards
the goal is to define these semantics. In this chapter, we will present our
published contribution~\cite{DasguptaAdve:PLDI19} of the most complete and
thoroughly tested formal semantics of \ISA to date.  Our semantics faithfully
formalizes all the non-deprecated, sequential user-level instructions of the
\ISA Haswell instruction set architecture.  This totals \currentIS{}
instruction variants, corresponding to \currentIntel{} mnemonics. %
%~\cite{IntelManual} (Section~\ref{sec:IC}).  
The semantics is fully executable and has been tested against more than 7,000
instruction-level test cases and the GCC torture test suite. % using the
co-simulation method, This extensive testing paid off, revealing bugs in both
the \ISA reference manual and other existing semantics.  \AEC{We also
  illustrate potential applications of our semantics in different formal
    analyses,
%such as symbolic execution, deductive verification, and translation
  %validation,
and discuss how it can be useful for processor verification.}

\chapter{Introduction}\label{sec:ba}

%The problem you want to tackle
Analyzing binary code is crucial in software engineering and security research.
Some of the notable applications of binary analysis can be found in binary
instrumentation
(\cite{Bruening:CGO2003,PEBIL10,Pin:2005,Valgrind:ENTCS03,DynamoRIO:2004}),
  binary translation~\cite{UQBT:2000}, software hardening
  (\cite{Cha:2015,Ford:2008,Zhang,Zhang:2013}), software testing
  (\cite{Chipounov:2011,Avgerinos:2014,godefroid_automated_2008}), CPU
  emulation (\cite{QEMU:USENIX05,Magnusson:2002}), malware detection
  (\cite{Christodorescu:2005,Kruegel:2004,BitBlaze:2008,BAP:CAV11,Egele:USENIX07,Yin:CCS07}),
  automated reverse engineering
  (\cite{Cui:2008,Lin:2008,Schwartz:2013,Yakdan2015NDSS,McSema:Recon14,Angr,Radare2}),
  sand-boxing~\cite{Kiriansky:2002:SEV,Erlingsson:2006,Yee:2009},
  profiling~\cite{Harris:2005,Srivastava:1994}, and automatic exploit
  generation~\cite{Cha:2012}.
               
                 Binary analysis is generally performed by various decompiler
                 projects
                 ~\cite{McSema:Recon14,Remill,Angr1,BAP:CAV11,Radare2}, whose
                 very first step is to translate the machine code to an
                 intermediate representation (IR), and thereby exposing many
                 high-level properties (like control flow, function boundary
                     and prototype, variable and their type etc.) of the
                 binary, which  assist further analysis and/or optimization.
                 Formally establishing faithfulness of the decompilation (i.e.
                     translation from machine code to high level IR) is pivotal
                 to gain trust in any binary analysis. Any bug in the
                 translation would invalidate the binary analysis results.  For
                 example, a malware analysis system might miss vulnerabilities
                 or a binary instrumentation system, instrumenting a buggy IR,
                 might lead to failure or even crash in interpreting the
                 instrumented program. Therefore, automatic validation tools
                 are needed urgently to uncover hidden problems in a binary
                 translator. 

% What is the current state of the art and why that is insufficient
Despite of the importance in establishing the faithfulness of the binary
translators, there has been surprisingly little effort towards that direction.
The most notable approaches are either based on hardware co-simulation
testing~\cite{Martignoni:ISSTA2009,Martignoni:ISSTA2010}, which is limited
because generating specific test-cases to uncover semantic bugs in a lifter is
non-trivial, or differential-testing~\cite{Martignoni:ASPLOS2012,ASE2017},
  which is limited in terms of coverage of the instructions validated and
  faithfulness guarantees it provides (Refer to
      Chapter~\ref{sec:related-work}). 

%How you plan to improve on the state of the art Given the importance  in
%establishing the faithfulness of binary lifter (or translator), 
We propose to employ \tv on binary lifters, as a means to establish the
faithfulness of the translation, by leveraging the semantics of the languages
involved (e.g. the Intel's X86-64 and the high-level IR).  Given the recent
advances in translation validation~\cite{Pnueli:1998} in validating
compilation~\cite{Necula:2000,Pnueli:1998,Stepp:2011,Tristan:2011,VOC2002,TVOC:CAV2005},
  employing \tv to validate binary translators seems a promising direction to
  explore,
%Given the recent advances in translation validation~\cite{Pnueli:1998} in
%validating
%compilation~\cite{Necula:2000,Pnueli:1998,Stepp:2011,Tristan:2011,VOC2002,TVOC:CAV2005},
  %a basic version of this strategy is likely quite feasible today,
  however, there are additional challenges to deal with when it comes to
  validating the decompilation pipeline (Refer Section~\ref{sec:challenges}).
  Moreover, we would like validation approach to be general enough and hence
  applicable to any state-of-the-art binary to high-level IR
  translators~\cite{McSema:Recon14,Remill,FCD,llvm-mctoll,BAP:CAV11,Angr1,DiFederico:CC2017}.
  The idea is to validate the translators without leveraging any knowledge of
  the translation involved, which in turn makes the problem even more
  challenging (Refer Chapter~\ref{sec:future}).
  %thereby avoiding any translator specific customization during validation
  %(Refer Chapter~\ref{sec:future}).  However, having this goal makes the
  %problem even more challenging.

We believe that formally validating the translation is more robust as compared
to (1) validating the translation using random (or low coverage) test-cases,
   and (2) differential testing technique which tests the correctness of a
   translation, generated by a translator, by comparing its behaviors with that
   provided by other translators under test. \cmt{This means, declaring a
     translation to be correct assumes the correctness of all the translator.}

%Summarize your contributions
\paragraph{Contributions}
Below we propose the  primary contributions of our work.

\textbf{\emph{Most Complete user-level \ISA Semantics:}} Towards the goals of
\tv of binary translators, we developed\cite{DasguptaAdve:PLDI19} the most
complete and thoroughly tested formal semantics of \ISA to date, which
faithfully formalizes all the non-deprecated, sequential user-level
instructions of the x86-64 Haswell instruction set architecture. 

\textbf{\emph{Translation validation on binary translators~}} We propose tools
and techniques to formally validate the translation from binary to high level
IR.  To the best of our knowledge, we are the first to propose \tv to establish
the faithfulness of binary lifters targeting LLVM as their high-level IR. We
would demonstrate the applicability of our approach by validating the
translation of a realistic decompiler, McSema~\cite{McSema:Recon14}.
%and revng~\cite{DiFederico:CC2017}}.

\textbf{\emph{Automated back-box approaches to \tv~}} We would like our
technique to work automatically and uniformly across translators considering
the translators as black-boxes, i.e. without using any translator-generated
hints  or translator-specific heuristics to assist the validation. We would
like to demonstrate the effectiveness  of our solution by validating against
two or more decompilers.  Optionally, we would like to use machine learning
techniques to infer the  variable or basic block correspondence between binary
and LLVM IR code which will help in realizing a translator-agnostic approach to
\tv.

%\textbf{\emph{Translator-agnostic solution~}}     

%In the next section, we elaborate on our motivation to establish faithfulness
%of binary translators.  Why it is important \section{Benefits of Binary
  %Analysis}

\section{Preliminaries}
\label{sec:prelim}

In this section, we provide background on various pieces of our work:
(i) The binary lifter under test, McSema, (ii) The formal \ISA semantics, and
(iii) The formal \LLVM semantics.

\paragraph{McSema}\label{par:mcsema} \mcsema~\cite{McSema:Recon14} is the most
mature, well tested, open-source lifter to raise binaries from \ISA
instructions to LLVM bitcode.  At a high-level, \mcsema is split into two
parts: (a) frontend, and (b) backend. The frontend is responsible for parsing,
loading, and disassembling a binary and exports an interface to the backend to
query for the required information, e.g., the defined symbols, sizes of
various binary sections, instruction listings etc. The backend then uses this
information and Remill~\cite{Remill} library to lift the individual
instructions. McSema supports multiple different frontends with IDA Pro being
the most robust, and supported option.

Conceptually, the backend implementation of \mcsema is fairly straightforward:
\mcsema exposes all of architecture state, i.e., the program registers,
conditional flags, and program memory, through an LLVM \emph{struct}, aptly
named \Mcstate. 
Member fields
of the structure correspond to every register (register name, not a physical
register) and flags that can be used during the execution of the program. Instructions
operating on the stack must retrieve the current top of stack from the
appropriate member field (corresponding to \reg{rsp}) in the structure.
\mcsema simply scans through the disassembly of the binary
and lifts each instruction one by one, emitting code to read and update the
members of \emph{state} as defined by the instruction semantics. In essence,
the code lifted by \mcsema simply \emph{simulates} the binary in \LLVM.

%The main idea behind \mcsema lies in the simulation of the original binary,
%which in turns requires architectural state and memory of the target
%processor. There is no explicit stack by default.  To simulate state of the
%processor, an LLVM structure type called \emph{state} is used. Member fields
%of the structure correspond to every register (register name, not a physical
%register) that can be used during the execution of the program. Instructions
%operating on the stack must retrieve the current top of stack from the
%appropriate member field (corresponding to \reg{rsp}) in the structure.
%%\todo[inline]{Unclear: Since structure only holds registers, say which
%%register is used to retrieve "current top of stack."}
%\mcsema  performs decompilation of the binary code by translating the machine
%instructions to operations on this \emph{state} structure, according to the
%machine specification. This lifting process works by translating each assembly
%instruction in the procedure into an IR sequence,  which explicitly specifies
%how it affects the machine's memory and registers, including the flags.
%\paragraph{\ISA \& LLVM formal semantics}
%The present work needs the formal
%semantics of \ISA and LLVM, which we borrowed from~\cite{DasguptaAdve:PLDI19}
%and~\cite{LLVMSEMA} respectively. Both semantics are developed using
%\K~\cite{k-primer-2013-v32}\cmt{\url{http://kframework.org}}, which is a
%framework for defining formal language semantics.

\paragraph{K-Framework}\label{par:k} The presented work needs the formal
semantics of \ISA and LLVM, which we borrowed from~\cite{DasguptaAdve:PLDI19}
and~\cite{LLVMSEMA} respectively (and described next). Both semantics are developed using
\K~\cite{k-primer-2013-v32}\cmt{\url{http://kframework.org}}, which is a
framework for defining formal language semantics. Given a syntax and a
semantics of a language, \K automatically generates a parser, an interpreter, 
a symbolic execution engine, as well as
formal analysis tools such as model checkers and deductive program verifiers,
at no additional effort. Using the interpreter, one can test their
semantics immediately, which significantly increases the efficiency of
semantics developments. Furthermore, the formal analysis tools
facilitate formal reasoning about the given language semantics.  This
helps both the applicability of the semantics and in the
engineering the semantics itself.

\paragraph{\ISA Formal Semantics} Our work uses the state-of-the-art \ISA
semantics developed by Dasgupta et al.~\cite{DasguptaAdve:PLDI19}, which
presents the most complete and thoroughly tested formal semantics of x86-64 to
date, and faithfully formalizes all non-deprecated, sequential
user-level instructions of x86-64 Haswell instruction set architecture.
This totals to 3155 instruction variants, corresponding to 774 mnemonics.
Their semantics are fully executable, and includes a symbolic execution engine
automatically generated by \K framework
%
    \footnote{Given a syntax and a semantics of a language, \K automatically
    generates a parser, an interpreter, a symbolic execution engine, as well
    as formal analysis tools such as model checkers and deductive program
    verifiers, at no additional effort.}.
%

\paragraph{\LLVM Formal Semantics} We use the LLVM formal semantics as defined
in \K~\cite{LLVMSEMA}, which models LLVM types (integers, composite arrays and
structs, corresponding pointers), the \texttt{getelementptr} instruction (used
to compute the address of an element nested within a composite),  integer
arithmetic \& comparison operators, memory operations (\texttt{load},
\texttt{store}, and \texttt{alloca}), control flow instructions for
unconditional and conditional branches, as well as function calls and returns.
However, the semantics does not support: floating point, vector types, and
most LLVM intrinsic functions and therefore we cannot validate the \tv for
such instructions.  This is a limitation of the available LLVM semantics, and
not a limitation of our work.

.
%\todo[inline]{The last 2 sentences are nice, but could be dropped if space is tight.}
%\todo[color=yellow]{These two semantics are two of the most
%    important building blocks. Can we make them top-level paragraphs?}
%\paragraph{\ISA Semantics} \todo{These two semantics are two of the most
%important building blocks. Can we make them top-level paragraphs?}
%The formal model~\cite{DasguptaAdve:PLDI19}
%presents the most complete and thoroughly tested formal semantics of x86-64 to
%date, which faithfully formalizes all the  non-deprecated, sequential
%user-level instructions of the x86-64 Haswell instruction set architecture.
%This totals 3155 instruction variants, corresponding to 774 mnemonics.  The
%semantics is fully executable, and comes with an
%automatically generated symbolic execution engine (thanks to the \K framework),
%which we leverage in this work.
%
%\paragraph{LLVM Semantics} The formal semantics of \LLVM is provided by
%~\cite{LLVMSEMA}, which models  various LLVM types (like integer types,
%    composite array and struct types, the  corresponding pointer types, and the
%    \texttt{getelementptr} instruction used to compute the address of an element nested
%    within a composite type),  integer arithmetic \& comparison operators,
%  memory operations (like \texttt{load}, \texttt{store}, and \texttt{alloca}), 
%  control flow instructions
%  for unconditional and conditional branches, as well as function calls and
%  returns. The current version of the semantics does not support floating point
%  or vector types and most LLVM intrinsic functions, which constrains our experiments
%  but does not affect the overall approach proposed and evaluation in this work.
%  \todo{Check last sentence.}

%\paragraph{Stoke Libraries}\label{par:stoke} Stoke~\cite{Stoke2013} is a
%stochastic superoptimizer and program synthesizer for the x86-64 instruction
%set.  The project comes with many useful libraries to (1) query interesting
%properties of \ISA instructions (like type, size, inputs \& outputs etc.),
%(2) disassemble \ISA programs, and (3) analyze  \ISA programs to infer
%properties related to control- \& data-flow. We make use of these in our
%work.
%
%The rest of this paper is organized as follows: we discuss our \siv
%in~\ref{sec:siv}, and then move on to show how we use the validated sequences of
%IR instructions to construct a \emph{\compd} and use it to for \plv
%in~\ref{sec:plv}. Finally, we show the effectiveness of our developed techniques
%through our evaluation in~\ref{sec:eval}.

\section{Formalization of \ISA Semantics}
\label{sec:harvestsema}
This section presents how we get the complete semantics of all the
user-level instructions. Section~\ref{sec:IC} details the scope of our work. Section~\ref{sec:Approach} mentions how we leverage the information available in \Strata, our baseline semantics. Section~\ref{sec:x86sema} explains how we formalize our model in \K.

\subsection{Scope of the Work}\label{sec:IC}
\SC{We support all but a few non-deprecated user-level instructions. The support includes  \currentIS{} total variants of the Haswell \ISA ISA (representing \currentIntel{} out of \totalIntel{} 
unique mnemonics)}. The entire implementation took 8 man-months, with the lead author having prior experience in binary decompilation and strong familiarity with the \ISA architecture and documentation.
% (not including extensive time spent on projects related to binary decompilation that gave the lead author relevant experience and strong familiarity with the \ISA architecture and documentation).  
Below is a summary of the instruction categories that we support.
%\begin{itemize}
%    \item \textbf{General-Purpose Instructions:} These implement data-movement, arithmetic, logic, control-flow, string operations (including repeated- and fast- string operations).
%    
%    \item \textbf{Streaming SIMD Extensions (SSE) Instructions \&   subsequent extensions (SSE-2, SSE-3, SSE-4.1, SSE-4.2):} Instructions in this category operate on integer, string or floating-point values stored in $128$-bit xmm registers. Among other things, the category features instructions related to conversions between integer and floating-point values with selectable rounding mode, and string processing.
%    
%    \item \textbf{Advanced Vector Extensions (or AVX) \& subsequent extensions (Fused-Multiply-Add (or FMA) \& AVX2):} These instructions operate on integer or floating-point values stored in $256$-bit ymm registers; a majority of which are promoted from SSE instruction sets. Additionally, the category features enhanced functionalities specific to AVX \& AVX2, like  broadcast/permute, vector shift, and non-contiguous data fetch operations on data elements. 
%    \item \textbf{$\textbf{16}$-bit Floating Point Conversion (or F16C):} These instructions implement conversions between single-precision ($32$-bit) and half-precision ($16$-bit) floating-point values. 
%\end{itemize}

\emph{General-Purpose.~}
These implement data-movement, arithmetic, logic, control-flow, string operations (including fast- and repeated- string operations).
    
\emph{Streaming SIMD Extensions (SSE) \& subsequent extensions (SSE-2, SSE-3, SSE-4.1, SSE-4.2).~}
Instructions in this category operate on integer, string or floating-point values stored in $128$-bit xmm registers. Among other things, the category features instructions related to conversions between integer and floating-point values with selectable rounding mode, and string processing.
    
\emph{Advanced Vector Extensions (AVX) \& subsequent extensions (Fused-Multiply-Add (FMA) \& AVX2).~}
These instructions operate on integer or floating-point values stored in $256$-bit ymm registers; a majority of which are promoted from SSE instruction sets. Additionally, the category features enhanced functionalities specific to AVX \& AVX2, like  broadcast/permute, vector shift, and non-contiguous data fetch operations on data elements. 

\emph{16-bit Floating-Point Conversion (or F16C).~}
These instructions implement conversions between single-precision ($32$-bit) and half-precision ($16$-bit) floating-point values. 

Instructions which are \emph{not included} in the current scope of work are: 
(1) System-level instructions, which are related to the operating system, 
protection levels, I/O, cache lines, and other supervisor instructions; 
(2) x87 \& MMX instructions, consisting of legacy floating-point and vector operations, respectively, which are now superseded by SSE; 
(3) Concurrency-related operations, including atomic operations and fences; and 
(4) Cryptography instructions, which support cryptographic processing specified by Advanced Encryption Standard (AES). 
%
\SC{We note that while there is no inherent limitation in supporting the above instructions with our approach, the system-level instructions require to formulate an abstraction of different architectures and operating systems, which is a significant effort that is orthogonal to the presented effort of formalizing the user-level instructions.}


%\SC{Note that, there is no inherent limitation in supporting system-level and cryptography instructions with our approach. However, the system-level instructions require to formulate an abstraction of different architectures and operating systems, which is an orthogonal, significant effort that we think should be deserved as an independent work, and we wanted to separate it from the presented effort of formulating the user-level instructions. Nevertheless, K framework makes it easy to add additional state components without modifying rules for operations that do not require those components. We expect our approach to work equally well compared with existing approaches, such as ~\cite{Goel:ProCoS17}, which implements a subset of system-level instructions. On the other hand, the cryptography instructions are omitted mainly because they are not given a high priority.}

\subsection{Overview of the Approach}
\label{sec:Approach:Overview}

Briefly, our approach is as follows.
%
We first defined the machine configuration and underlying infrastructure in the \K framework, in order to define, execute and test the \ISA semantics.
%
\SC{To leverage previous work as much as possible, we took the semantic rules of all the instructions supported in Strata, which amounts to about 60\% of the instructions in scope, in the form of SMT formulas.}
%
We corrected, improved or simplified many of the baseline rules.
%
We then translated these SMT formulas from Strata into \K rules using a script, and tested the resulting rules by comparing with the Strata rules using Z3.
%
These steps give us a validated initial set of semantic rules in \K for about 60\% of the target instructions (our ``baseline'' set).

We attempted to extend the stratification approach in Strata to define additional rules automatically, in two ways: (i) augmenting their base set \s{B}, and (ii) constraining the search space manually using knowledge of instruction behaviors.  Both these attempts failed; they worked only for a few instructions, but in general, we found them to be impractical. Specifically, we added $58$ base instructions to the base set, and learned the semantics of $70$ new instructions, which are variants of the added  instructions, in $20$ minutes, but no more even after we kept running for two days. Also, we tried constraining the search space by manually populating it with relevant instructions. The lesson we learned from these experiments is, getting the right set of base instructions or a constrained search space for a complex instruction need an insight about the semantics of that instruction itself. We found that the effort to extract such information from the manual is about the same  as manually defining that instruction.
 

%\Comment{SANDEEP: Please check this last sentence and fix / improve.}

We then manually added \K rules for the remaining 40\% of the target instructions by systematically translating their description of the Intel manual into \K rules, in some cases cross-referencing against semantics available in Stoke.
%
The outcome was a complete formal specification of user-level \ISA in \K.

We validated this semantics in three ways, as described in Section~\ref{sec:Eval}.
%
First, we use the \K interpreter to execute the semantics of \emph{each} instruction for 7,000+ test inputs (each input is a processor state configuration) and compared the output directly with the hardware behavior for the same instruction.
%
Second, we repeated this experiment using the applicable programs in the \SC{GCC C-torture tests~\cite{CTORTURE}}.  % (\TortureInclude{} out of \TortureTotal{} tests).
%
Third, we compared against the semantics defined in the Stoke project for about 330 instructions that were omitted in Strata (thus not included in our baseline), using an SMT solver.

These validation experiments uncovered bugs in the Intel manual, in Strata's simplification rules, and in the Stoke semantics.  All these bugs were reported to the authors, and most have been acknowledged and some have been fixed.  The details are in Section~\ref{sec:Eval}.



\input{formal-X86-semantics/kx86.tex}
\section{Evaluation} \label{sec:Eval}

\paragraph{Instruction Level Validation}

\subparagraph{Results}

\subparagraph{Inconsistencies Found} 

\paragraph{Program Level Validation}
\input{formal-X86-semantics/application.tex}
\section{Limitations}\label{sec:limit}

\section{Related Work}\label{sec:RW}

% \Qt{Suggestion: I do not think we should include coverage numbers of projects that do not support direct semantics. And here are the reasons:
%     1. Projects like angr or mcsema do have support  of x87 or mmx which we do not have. That way if we include 
%     x87/mmx in the total instructions count we cannot say that we are 100\%.
%     2. Now if we exclude x87/mmx instructions and give the percentage (and that way we can say we are 100\%), but  that may confuse people. 
% 
%     We can include Strata and Goel's work in the table (or write in text ) and *can* include the percentages because neither of them support x87/mmx. For the others we can argue that they are not direct. Also we may 
%     add that bap, remill and radare2 are not complete. I can later will in what exact instruction they are missing  
%  }
% 
% \Qd{\revisit{I think we should ommit exact percentage numbers all together in the table. We should replace that column with a simple yes/no checkmark whether the semantics is complete. Also remove all together projects that do not give semantics directly to x86 but to some IL.}}

%There have been many projects that host a formal semantics of \ISA either as
%their main contribution or as part of their infrastructure.
%Table~\ref{table:RW} summarizes such previous work and compares it to our formal semantics.\footnote{Here we focus on comparing with other \emph{direct} semantics, since a complete \emph{direct} semantics is our goal and required for our purpose. We will discuss other \emph{indirect} semantics later in this section.}
%We do the comparison in three directions that reflect the
%primary contributions of our work: the completeness of the definition in terms
%of supported user-level instructions, the faithfulness of the definition in
%terms of whether it is executable and hence can be evaluated with real code
%execution, and the generality of the definition in terms of its applicability to
%formal reasoning analyses. Next, we discuss in more detail each of the
%related works.

There have been many projects that host a formal semantics of \ISA either as
their main contribution or as part of their infrastructure.
This section summarizes such previous work and compares it to our formal semantics based on three directions that reflect the primary contributions of our work: completeness, in terms of supported user-level instructions; faithfulness, in
terms of whether it is executable and hence can be evaluated with real code
execution; and generality, in terms of its applicability to
formal reasoning techniques.

\cmt{
\begin{table}
\scalebox{0.7}{
%\setlength{\arrayrulewidth}{.15em} 
\begin{tabular}{l||ccc}
\hline
\\ [-10pt]
\multicolumn{1}{l||}{\begin{tabular}[]{@{}c@{}}Project \\ Name\end{tabular}} & 
\multicolumn{1}{c}{\begin{tabular}[c]{@{}c@{}}Complete Support of \\ \ISA User-Level \\ Instructions (in scope) \end{tabular}} &
\multicolumn{1}{c}{\begin{tabular}[c]{@{}c@{}}Executable \\ Semantics\end{tabular}} &
\multicolumn{1}{c}{\begin{tabular}[c]{@{}c@{}}Support for \\ Full-Fledged \\ Formal Reasoning\end{tabular}} \\
\hline 
\\ [-10pt]
Strata~\cite{Heule2016a}          & \xmark & \rating{50} & \xmark      \\
Goel et al.~\cite{Goel:FMCAD14}  & \xmark & \cmark      & \cmark      \\
CompCert~\cite{Leroy:2009}        & \xmark & \cmark      & \cmark      \\
%Remill~\cite{Remill}              & \xmark & \cmark      & \xmark      \\
TSL~\cite{TSL:TOPLAS13}           & \xmark & \cmark      & \rating{50} \\
Sail ~\cite{sail-framework}        & \xmark & \cmark      & \cmark \\
Roessle \etal~\cite{Roessle:CPP19} & \xmark & \rating{50} & \cmark \\
\hline
\textbf{Our Semantics}            & \cmark & \cmark      & \cmark      \\
\hline
\end{tabular}}
\begin{center} 
{\small
    \cmark : Yes
    \quad
    \xmark : No % due to incorrect semantics
    \quad
    \rating{50} : Partially True
    \hfill
    %\quad
    %\emph{NJ}: Node.js 0.10.29
}
\end{center}
\caption{Projects hosting formal semantics of the \ISA ISA.}
\label{table:RW}
\end{table}
}

\Strata~\cite{Heule2016a} uses program synthesis to generate the instruction
semantics of X86-64 as SMT bit-vector formulas\cmt{, by learning their input/output behavior
through execution on an actual processor}. Automatically learning the formal semantics of 60\% of the target \ISA ISA
is impressive, and we leverage this result in our work.  However, the other 40\% of the
user-level instructions are not straightforward to automatically learn by their algorithm, mainly due to limitations of the underlying synthesis engine.  Moreover, the specifications are executable only for non-floating-point (FP) instructions.
%The FP operations are represented in the SMT formulas of the definition as
%uninterpreted functions. 
%Finally the specifications are given as SMT formulas but 
% have not been demonstrated to be usable in a formal analysis setting out-of-the-box.

\SC{A contemporary work by Roessle \etal~\cite{Roessle:CPP19} presents a method to extract the big step semantics of a binary program using the small step instruction semantics extracted mostly from Strata\footnote{There are some minor omissions on immediate instructions with $8$-bit operands for which Strata learns $256$ brute force formulas.} plus some manually drafted support for branching instructions and stack operations. Like Strata, their specification is executable only for the non-floating-point instructions. Moreover, their work does not aim for completeness of semantics, one of our primary goals.}

Goel \etal~\cite{Goel:FMCAD14} use the ACL2 theorem prover~\cite{ACL2:Kaufmann2000} to model the \ISA ISA and they support
\goelPerc{} of all user-level instructions~\cite{GoelList}, plus some system-level instructions, paging, and
segmentation.  This list is far from a complete semantic definition of \ISA,
but it is still the state-of-the-art in terms of formal analysis applied
directly to \ISA code. It is also an executable definition as demonstrated by
its use for simulations. In our work, we do not leverage this definition, since
\Strata has defined many more instructions.

The CompCert verified compiler~\cite{Leroy:2009} includes semantics
definitions for all intermediate and target languages used within the compiler,
including a definition for 32-bit x86 assembly. The definition is specified in Coq~\cite{Coq} and has been used in a formal
setting for proving the correctness of CompCert's compilation step to assembly,
as well as outside CompCert, e.g., in proofs relating to the certified concurrent
OS kernel CertiKOS~\cite{Gu:2016}. However, this definition focuses on the
32-bit x86 instruction set, which is a subset of the \ISA instruction 
set.
Moreover, it is part of the trust base for CompCert and it is not clear
whether or how it has been tested against an actual processor, whereas
\Strata and ours have been extensively tested.

TSL~\cite{TSL:TOPLAS13} is a system that can auto-generate tools for
various machine code analyses given a semantics definition of the machine
language written in the TSL specification. Such a semantics
definition for the integer instructions (i.e., no floating-point instructions) of the $32$-bit x86 instruction set is given
as part of the project. It is used to generate
various tools, including a machine code synthesizer~\cite{Srinivasan2015}.
This definition, to our knowledge, has not been used
for formal verification proofs, i.e., to prove whether a given x86
program meets its specification.

Our semantics, like all the other work cited above, uses a sequential consistency memory model, and not weaker memory models.
Existing efforts to specify weaker memory models for \ISA such as Owens et al.~\cite{Owens:x86-TSO} and Sarkar et al.~\cite{Sarkar:POPL09}, however, suffer from their limited support for instruction semantics (i.e., they consider only a small subset of 32-bit x86 instruction set).
We believe that integrating these two complementary efforts is a promising direction toward rigorously reasoning about real-world programs running on modern multiprocessors (e.g., using the Sail framework as we will describe below).

\SC{%
Sail is another language semantics framework, tailored for describing an instruction-set architecture semantics.  Sail has been used to specify the semantics of ARMv8-A, RISC-V, and CHERI-MIPS~\cite{sail-popl2019}, as well as the semantics of a small subset of x86~\cite{sail-x86}.  Sail is similar to the K framework we employed, but K is far more general-purpose than Sail.  Also, the Sail x86 semantics is much more limited than ours.  It describes the semantics of a fragment of 32-bit user-mode x86 instructions, while ours covers also the 64-bit counterpart as well as the floating-point instructions.
%
Sail, however, allows us to integrate a semantic definition with their relaxed memory models~\cite{rmem, Pulte:2017} for concurrency semantics.  We believe that (automatically) translating our semantics into Sail\footnote{Indeed, the Sail ARMv8-A semantics is automatically generated from the ARM-internal specification of ARMv8-A~\cite{Reid2017} written in the ARM's architecture specification language, ASL~\cite{asl}, by using the ASL-to-Sail translator~\cite{sail-popl2019}.} is a promising direction to obtain concurrency semantics and thus enable concurrency reasoning for x86 programs, which we leave as future work.
}

\SC{Overall, the key differentiator of our effort compared to the existing work, as cited above, is that our semantics achieves (A) completeness of supported  user-level instructions, (B) faithfulness, and (C) applicability to formal reasoning analyses. In Section~\ref{sec:lesson-learned}, we elaborate on our novel approaches that allow us to achieve this unique combination.}

There are various binary analysis projects that target \ISA binaries
and lift them to a higher-level representation more suitable for the
specific analysis. These include Angr~\cite{Angr} using the VEX IR of Valgrind~\cite{Valgrind:ENTCS03}, the QEMU~\cite{QEMU:USENIX05} emulator
using the TCG IR, the software fault isolation tool RockSalt~\cite{Roclsalt:PLDI12} using its own RTL DSL, the disassembler and binary analyzer Radare2~\cite{Radare2} using the ESIL IR~\cite{ESIL}, the binary analysis
tool BAP~\cite{BAP:CAV11} using the BIL IR, and  the static binary translator Remill~\cite{McSema:Recon14} using LLVM IR~\cite{LLVM:CGO04}.
We refer to these semantics as \emph{indirect} because they give the semantics of the \ISA binary via the translation to their IR, as opposed to a \emph{direct} semantics such as ours and the others cited earlier.
% because in all these cases the lifted IR, that is being analyzed, is rigorously defined as opposed to providing \emph{direct} semantics to the binary.
A direct semantics has significant advantages over an indirect semantics.
For example, without the direct semantics of \ISA, we cannot even formulate the correctness of a translator from \ISA to the IR.
Analogously, many programming languages (C, C++, Java, etc.) have been given direct semantics, instead of indirect semantics by translation to other languages, for formal reasoning at the desired language granularity. 

% Even though tools like BAP and Angr can do some formal reasoning owing to their capability of symbolically executing the IR semantics, but they are not designed with the goal of full-fledged formal reasoning.



\cmt{ 
Regarding the comparison with previous work, we focused on comparing with other direct semantics, since a complete *direct* semantics is our goal and required for our purpose.

In all these
cases the IR that is being analyzed is rigorously defined, but we refrain from
considering these as formal specifications of \ISA because the
actually specified language has abstracted away various features of \ISA.
For example, both VEX IR and RockSalt use different simplified register
semantics: VEX IR omits many implicit bit truncations
and/or extensions that are part of many \ISA instruction semantics 
(i.e., these have to be emulated separately by the program), while
RockSalt's DSL uses an infinite register file instead of the finite
\ISA register file.
}

Hasabnis \etal~\cite{Hasabnis:ASPLOS16, Hasabnis:FSE16} also present an indirect semantics of \ISA, but in contrast to other indirect semantics, they use machine learning~\cite{Hasabnis:ASPLOS16} and symbolic 
execution~\cite{Hasabnis:FSE16} to automatically learn the translation of \ISA instructions to their IR, by extracting knowledge from the hard-coded  translation logic of compilers such as GCC.
However, as they admitted~\cite{Hasabnis:FSE16}, their semantics omits some important details of \ISA semantics (e.g., the effect of various instructions on CPU flags), and thus is not sufficient to serve as a solid foundation for rigorous formal analyses of \ISA binary.

  
\cmt{ 
Remill~\cite{McSema:Recon14} is a static binary translator from \ISA to LLVM
IR~\cite{LLVM:CGO04}. The translator contains specifications for \ISA instructions
in the form of equivalent C\cmt{LLVM IR} programs to assist the translation. This
specification is neither complete nor formal and cannot be easily used
in formal analysis.
}


% implement
% lifting of \ISA to an architecture-independent intermediate language (IL).
% In contrast with the other works above,

% They learn these mappings by extracting
% knowledge from the hard-coded translation logic found in compilers such as GCC.
% The extracted mappings cover more than 99\% of \ISA instructions. However,
% \cmt{similar to the rest of binary analysis works,}
% the resulting IL cannot stand as a formal \ISA definition because it
% abstracts away important \ISA semantic details, e.g.,
% the effect of the various instructions on CPU flags.




\section{Lessons Learned}
\label{sec:lesson-learned}
Here we present the lessons we learned during our semantics development,
     identifying important aspects to be considered, and clarifying best
     practices for developing a large ISA semantics.  We also discuss the
     novel aspects of our semantics development approach that allow us to
     obtain a complete and faithful semantics with a practical amount of
     effort.

\paragraph{Automatic semantics synthesis}

Most previous efforts in formalizing \ISA semantics can be categorized based on whether the underlying approach is fully manual ~\cite{Goel:FMCAD14, TSL:TOPLAS13, Leroy:2009, sail-x86} or fully automatic~\cite{Heule2016a, Roessle:CPP19, Hasabnis:ASPLOS16, Hasabnis:FSE16}. We note that none of these approaches, when used in isolation, sufficiently scale to a complete and faithful semantics, as much as ours that combines these complementary approaches so that they benefit from each other.

Section~\ref{sec:Approach:Overview} reports the challenges we encountered in achieving fully automatic synthesis of the entire \ISA semantics.
Specifically, in a vast instruction set like \ISA \cmt{computer (CISC) architecture such as x86}, it is common that many instructions can be grouped together where the instructions of each group are similar to each other except for a few differences.
%One approach to automatic semantics synthesize is to employ stratification approach~\cite{Heule2016a} to learn the semantics of instruction variants in a group using the representatives of the group.
An automatic synthesis technique leveraging such a group, such as the stratification approach~\cite{Heule2016a}, would effectively synthesize such instruction variants' semantics, provided that the semantics of representative instructions in each group are given in advance.\footnote{For certain complex instructions, the size of their group is very small (i.e., they are quite different to each other), and thus the automatic synthesis would not yield a sufficient gain over the effort of specifying the semantics of their representatives, but we found that the number of such isolated instructions of x86-64 is small.}
The problem, however, is that it is non-trivial to properly partition all the instructions into such groups, providing the representative instruction semantics for each group, \emph{without} a priori knowledge about the semantics of all instructions.
%The stratification approach~\cite{Heule2016a} had been proposed to solve this dilemma, but it turned out to be not sufficient, leaving a substantial part of semantics unspecified.
The vanilla stratification approach~\cite{Heule2016a} turned out to be not sufficient to solve this dilemma, leaving a substantial part of the semantics unspecified.
Thus, we decided to manually provide the information about the partition and representatives, for which we had to consult the manual to obtain knowledge about the remaining part of semantics.
Once we obtained the required knowledge, however, we realized that it would be more straightforward to directly turn the knowledge into the semantics than going through the synthesis process, and thus we ended up manually specifying the remaining part of the semantics.

%As mentioned in Section~\ref{sec:Approach:Overview}, we found that our ideas of extending \Strata do not scale because we need information about the instructions, whose semantics we want to learn, in order to constraint the search space. This lesson suggested a new idea: the information needed to reduce the search space can be automatically extracted from the manual. 
%This  information does not need to be precise, and we believe that such rough information can be automatically extracted from the manual using text processing.  We plan to explore this idea while defining the semantics of unspecified and/or new instructions.

%As an alternative, we suggest a hybrid synthesis approach that we believe is practically promising, in particular for complex instruction set computer (CISC) architectures such as x86.
%
%While it is labor-intensive and error-prone to manually specify all the instruction variants, the automatic synthesis technique is shown to be effective to synthesize such instruction variants' semantics, provided that the semantics of its representative instruction is given in advance.
%
%Thus, in a hybrid approach, the semantics of representative instructions are manually written, and their variants are automatically synthesized. We believe that this hybrid approach is in a sweet spot, where the machine's search power and the human reasoning are effectively combined.
%We plan to apply and evaluate this idea when we specify the semantics of new instruction that will be introduced in the next version of x86-64 ISA.

Another important step of the semantics synthesis is post-processing. The generated semantics is often verbose and not necessarily human-readable. The post-processing step is desired to simplify the generated semantics to be succinct, which helps to increase the human-readability as well as to improve the efficiency when being employed in other applications (e.g., the size of SMT formula encoding can be reduced, which can reduce the burden of SMT solvers). For our semantics development, we have written dozens of simplification rules that are fed to the \K framework to simplify the synthesized semantics further (Section~\ref{sec:Approach}). 

\paragraph{Modeling and executing implementation-dependent behaviors}

The \ISA ISA standard admits \emph{implementation-dependent behaviors} for
certain operations on certain input patterns, that is, each processor
implementation can freely choose the execution behavior for each such case
(Section~\ref{sec:challenges-in-formalizing-x86}).
%
Faithfully modeling the implementation-dependent behaviors is necessary for the correctness of the
semantics.  For example, as mentioned in Section~\ref{subsec:compare-stoke},
  Stoke~\cite{Stoke2013} does not faithfully model such behaviors, causing
  certain errors in their semantics that we revealed~\cite{BugStoke986}.

There are two natural, faithful ways of specifying implementation dependent
behaviors.  One is to parameterize the semantics over the
implementation-dependent behaviors, and later instantiate it with a profile
that describes specific behaviors taken by the processor of interest. This
approach is desirable for validating the semantics using concrete execution.
Another is to introduce non-determinism in the semantics, which captures a set
of different possible behaviors in a single semantics, which is desirable
during symbolic interpretation of the ISA code.  We note that most of other
existing \emph{direct} \ISA semantics employ approaches similar to the ones
described above, faithfully modeling the implementation-dependent behaviors.
For example, Goel \etal~\cite{Goel:ProCoS17} models such behaviors using a
constraint function which is guaranteed to be unique and non-deterministic,
           while they employ the aforementioned profile-based approach for
           concrete execution.  TSL~\cite{TSL:TOPLAS13} makes both approaches
           available, from which their users can choose.

In our semantics, we faithfully modeled the undefined value as a unique symbol
(called undef) whose value is non- deterministically decided each time within
the proper range. For validating the semantics, we concretely executed the
semantics while the non-deterministic behaviors are represented symbolically
using the undef symbol and then we checked if the hardware output is matched by
(an instance of) the simulated output.

%The \ISA ISA standard admits \emph{implementation-dependent behaviors} for certain operations on certain input patterns, that is, each processor implementation can freely choose the execution behavior for each such case (Section~\ref{sec:challenges-in-formalizing-x86}).
%There are two natural, faithful ways of specifying implementation-dependent behaviors.
%One is to parameterize the semantics over the implementation-dependent behaviors, and later instantiate it with a profile that describes specific behaviors taken by the processor of interest.
%Another is to introduce non-determinism in the semantics, which captures a set of different possible behaviors in a single semantics.
%We took the second approach in this work since the implementation-dependent behaviors of x86 are quite limited and mostly localized (e.g., as in the example in Figure~\ref{fig:andn-semantics}).
%
%However, the non-determinism makes it non-trivial to execute (and validate) the
%semantics.  For example, during hardware co-simulation, the output of hardware
%execution may vary depending on the underlying processors, while the
%non-deterministic semantics execution will randomly choose certain behavior,
%  which may be different from the specific behavior implemented in a processor.
%  To mitigate this issue, for the hardware co-simulation, we symbolically
%  executed the semantics where the non-deterministic behaviors are represented
%  symbolically, so that the symbolic output captures all possible behaviors,
%  and then we checked if the hardware output is matched by (an instance of) the
%  symbolic output.
%
%Faithfully modeling the implementation-dependent behaviors is necessary for the correctness of the semantics.
%For example, as mentioned in Section~\ref{subsec:compare-stoke}, Stoke~\cite{Stoke2013} does not faithfully model such behaviors, causing certain errors in their semantics that we revealed~\cite{BugStoke986}.
%On the other hand, most of other existing \emph{direct} \ISA semantics employ an approach similar to the ones described above, faithfully modeling the implementation-dependent behaviors.
%For example, Goel \etal~\cite{Goel:ProCoS17} models such behaviors using a constraint function which is guaranteed to be unique and non-deterministic, while they employ the aforementioned profile-based approach for concrete execution.
%TSL~\cite{TSL:TOPLAS13} makes both approaches available, from which their users can choose.
%
%%\Comment{Instead of dropping the paragraph about the extra difficulty of using the non-determinism strategy, you need to discuss the tradeoffs. So describe both the benefits and the drawbacks of the non-determinism strategy. I'm just uncommenting the old text here, for now.}
%%%
%%Note that the non-determinism strategy makes it non-trivial to execute (and validate) the semantics.
%%For example, in the hardware co-simulation, the output of hardware execution may vary depending on the underlying processors, while the non-deterministic semantics execution will randomly choose certain behavior, which may be different from the specific behavior implemented in a processor.
%%To mitigate this issue, for the hardware co-simulation, we symbolically executed the semantics where the non-deterministic behaviors are represented symbolically, so that the symbolic output captures all possible behaviors, and then we checked if the hardware output is matched by (an instance of) the symbolic output.

\paragraph{Employing multiple semantic engineering frameworks}

We found that employing multiple semantic frameworks is helpful. Specifically, we employed the two semantic frameworks, \K and Stoke, where we enjoyed all of their (executive) benefits that make it easier for us to write and validate the semantics, and utilize the semantics in various applications. For example, we wrote the semantics of certain complicated instructions (e.g., \instr{pcmpestri}, \instr{pcmpestrm}, and \instr{pclmulqdq}) in \K, as \K provides an easy way to specify behaviors with multiple cases, while Stoke would have required us to write a big nested if-then-else expression, which is not convenient. As another example of the benefits, we used Stoke to validate most of our instruction semantics
as Stoke provides an infrastructure\footnote{Indeed, we contributed to their infrastructure as well~\cite{completing-stock,improving-stoke}.} for hardware co-simulation, whereas we employed \K to validate the semantics of floating-point instructions as Stoke does not support executing floating-point operations while \K does.

In order to use the two frameworks interchangeably, we developed a translator between the semantics of the two frameworks. To check the correctness of the translation, we verified equivalence between the original and the translated semantics for each instruction using the Z3 SMT solver.

To summarize, employing multiple frameworks with validated translation between them improved both the ease of specification (using \K) and ease of validation (using Strata), which expedited our semantics development process and thus significantly contributed to the completeness of our semantics. Moreover, we immediately benefit from all of their formal analysis tools, increasing the applicability of the semantics in various formal reasoning tasks. Existing semantics development efforts (e.g., \cite{Goel:FMCAD14,Heule2016a}), however, %(~\cite{Goel:FMCAD14,TSL:TOPLAS13,Leroy:2009,sail-x86,Heule2016a}) 
%use a single framework and do not obtain these important benefits.
employ a single framework without utilizing the potential of other frameworks, which otherwise might have improved completeness and/or faithfulness of their semantics with the same amount of effort.
%
%\cmt{
%We note that employing \Strata enables us to explore the near-completeness of its formalism, whereas \K framework provides many out-of-the-box formal analysis tools. The fact that we can use both the frameworks  interchangeably paid us off towards completeness and generality (in terms of using the semantics for formal analyses) of the formalism. On the other hand, other approaches (~\cite{Goel:FMCAD14,TSL:TOPLAS13,Leroy:2009,sail-x86})  did not explore this mutual benefit to its full potential\footnote{That, for some of these projects,  could be attributed to the fact getting a complete formal semantics is not the primary goal.}.

%%\section{Conclusion and Future Work}\label{sec:conc}
\section{Conclusion}\label{sec:conc}
We have presented the most complete formal semantics of \ISA user-level instructions
to date, and have thoroughly tested it using synthesized test inputs and the GCC torture tests.
%in different formal analyses such as symbolic execution, deductive verification, and translation validation.
We have also illustrated several potential uses of the semantics which are realized by the formal analysis tools
derived right from the \K specification. 
%Using the \K framework for our specification automatically provides several formal analysis tools from the specification, which greatly simplified these applications.
The \K framework also enables us to represent a semantics as SMT theories,
which other projects can
leverage for their own purposes.
%We describe several practical lessons we have learned from our experience in developing the semantics, which could be useful for future formal specifications of processor ISAs.


%As mentioned in Section~\ref{sec:Approach:Overview}, we found that our ideas of extending \Strata do not scale because we need information about the instructions, whose semantics we want to learn, in order to constraint the search space. This lesson suggested a new idea: the information needed to reduce the search space can be automatically extracted from the manual. This  information does not need to be precise, and we believe that such rough information can be automatically extracted from the manual using text processing.  We plan to explore this idea while defining the semantics of unspecified and/or new instructions.

%\SC{We plan to test our model against various x86-64 implementations (like AMD), which could uncover flaws in those implementations and/or additional imperfections in the manual.}

%\SC{Also, we aim to use our semantics for translation validation of the entire LLVM compiler back-end for X86-64, i.e., hundreds of thousands of lines of C++ code. Scalability is achieved by the modularity of the translation validation technique, where the verification of (small) sub-components are verified individually (hence can be massively parallelized) and their results are combined to obtain the final claim about the entire system.}

 

%\K  has success stories about defining the formal semantics of production languages like C~\cite{KC} and LLVM~\cite{KLLVM}. We can use that towards the benefit of binary lifters like ~\cite{McSema:Recon14, FCD} which lifts the binary code to LLVM IR for binary analysis and/or optimization. Having the semantics of both the source (\ISA) and the target language(LLVM) help in verifying the translation using symbolic reasoning and hence enhance trust in  the translation.

%\textit{Acknowledgements.}
%We warmly thank the \K team
%   for their technical support throughout the project.  Also we thank the Strata
%   developers for promptly confirming
%   our reported bugs and for answering all our questions in great detail.
   

